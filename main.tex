\documentclass{article}
\usepackage{graphicx} % Required for inserting images
\usepackage{amsmath}
\usepackage{xcolor}

\title{panel method notes}
\author{Jiayao Yan}
\date{November 2023}

\begin{document}

\maketitle

\section{
Source influenced potential computation
}

Source influenced potentials

\begin{equation}
\begin{aligned}
    \Phi_{\text{source}} = -\frac{1}{4\pi} \sum_{\text{panel edges}} \Bigg\{ &
        \frac{(x - x_a)(y_b - y_a) - (y - y_a)(x_b - x_a)}{d_{ab}} 
        \log \left( 
            \frac{r_a + r_b + d_{ab}}{r_a + r_b - d_{ab}} 
        \right) \\
    & - z \colorbox{yellow}{$\displaystyle \left[ 
        \tan^{-1}\left( \frac{m_{ab}e_a - h_a}{zr_a} \right) 
        - \tan^{-1}\left( \frac{m_{ab}e_b - h_b}{zr_b} \right) 
    \right]$} \Bigg\}
\end{aligned}
\end{equation}

Source terms

Doublet influenced potentials

\begin{equation}
    \Phi_{\text{doublet}} = \frac{1}{4\pi} \sum_{\text{panel edges}} \left[ \colorbox{yellow}{$\displaystyle \tan^{-1}\left( \frac{m_{ab}e_a - h_a}{zr_a} \right) - \tan^{-1}\left( \frac{m_{ab}e_b - h_b}{zr_b} \right)$} \right]
\end{equation}

Thus

\[
\Phi_{\text{source}} = -\frac{1}{4\pi} \sum \text{sources} \ \text{terms (not highlighted)} + z \Phi_{\text{doublet}} 
\]

Notes:

1. We combine the two $\tan^{-1}$ to avoid numerical issues and use the half-angle rule to change from $\text{atan2}$ to a function of $\text{atan}$.

2. For lifting surfaces, we can always just compute doublet\_ic\_p\_name first, and use declare\_variable in SourcePotential class to get the part. For non-lifting surfaces, we need to compute the doublet part first and then compute the whole source-induced potential. Note that even if there's no doublet element on the panel for non-lifting surfaces, these are just coefficients that happen to be the same.


\end{document}
